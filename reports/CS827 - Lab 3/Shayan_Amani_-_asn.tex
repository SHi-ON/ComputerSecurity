

\documentclass[11pt]{article}

\input{structure.tex} % Include the file specifying the document structure and custom commands

%---------------------------------
%	PACKAGES AND OTHER DOCUMENT CONFIGURATIONS
%---------------------------------

\usepackage{inconsolata}

\usepackage{algorithm}
\usepackage{algorithmic}

%% lst listing begin
\usepackage{listings}
\usepackage{xcolor}

\definecolor{codegreen}{rgb}{0,0.6,0}
\definecolor{codegray}{rgb}{0.5,0.5,0.5}
\definecolor{codepurple}{rgb}{0.58,0,0.82}
\definecolor{backcolour}{rgb}{0.95,0.95,0.92}

\lstdefinestyle{mystyle}{
    backgroundcolor=\color{backcolour},   
    commentstyle=\color{codegreen},
    keywordstyle=\color{magenta},
    numberstyle=\tiny\color{codegray},
    stringstyle=\color{codepurple},
    basicstyle=\ttfamily\footnotesize,
    breakatwhitespace=false,         
    breaklines=true,                 
    captionpos=b,                    
    keepspaces=true,                 
    numbers=left,                    
    numbersep=5pt,                  
    showspaces=false,                
    showstringspaces=false,
    showtabs=false,                  
    tabsize=2
}

\lstset{style=mystyle}
%% lst listing end

\newcommand{\NIL}{\mathrm{NIL}}


%---------------------------------
%	ASSIGNMENT INFORMATION
%---------------------------------
% Required
\newcommand{\assignmentQuestionName}{} % The word to be used as a prefix to question numbers; example alternatives: Problem, Exercise

\newcommand{\assignmentDueDate}{Wednesday,\ May 13\ ,\ 2020} % Due date
\newcommand{\assignmentClass}{CS 827} % Course/class
\newcommand{\assignmentTitle}{Lab 3} % Assignment title or name
\newcommand{\assignmentAuthorName}{Shayan Amani (sa1149)} % Student name
%\newcommand{\assignmentClassInstructor}{PROFESSOR NAME} % Instructor name/time/description

%---------------------------------

\begin{document}

%---------------------------------
%	TITLE PAGE
%---------------------------------

\maketitle % Print the title page

\thispagestyle{empty} % Suppress headers and footers on the title page

\newpage

%---------------------------------
%	REPORT
%---------------------------------

\section{Introduction}
I have used a linux server running CentOS 7 x64 little-endian with gcc (GCC) 4.8.5 20150623 (Red Hat 4.8.5-39) installed.

I have finally created the file \texttt{attack.ver} which has 16 random characters plus 8 more bytes to reach the pointer and finally adding the address of \texttt{rip} pointer of \texttt{fullversion()} to the end of file. The address is written in little-endian form as the machine's CPU architecture is.

\section{Experiment}

\textbf{1)} I have first turned off ASLR on the machine:

\includegraphics[width=1\columnwidth]{1-aslr.png}

Then I configured the following \texttt{makefile} for this lab with the appropriate targets:

\includegraphics[width=1\columnwidth]{1-makefile.png}

\textbf{2)} The buffer overflow happens in \texttt{strcpy()}
, at line 7, since strcpy is non-safe version of string copy functions in C and the safe counterpart is \texttt{strncpy()}. The reason for the overflow is that the function does not check the length of the stting to be copied to/from so an attacker can write as many bytes as they want to the buffer which overwrite other adjacent bytes and could lead to executing and gaining unauthorized permissions. 

\includegraphics[width=1\columnwidth]{2-source.png}]

\textbf{3)} I have created and tried a set of input files to study and figure out the behavior of the source code in terms of memory addresses and pointes. You can see them below:

\includegraphics[width=1\columnwidth]{3-full.png}
\includegraphics[width=1\columnwidth]{3-trial.png}
\includegraphics[width=1\columnwidth]{3-attack}

\textbf{4)} The memory dump of various functions in the source file are as follows:

\includegraphics[width=1\columnwidth]{4-main.png}
\includegraphics[width=1\columnwidth]{4-chk.png}
\includegraphics[width=1\columnwidth]{4-full.png}
\includegraphics[width=1\columnwidth]{4-trial.png}

\textbf{5)} In order to get the address of \texttt{fullversion()} function, I have acquired the needed stack information and states by running \texttt{info frame} command in \texttt{gdb}. You can see the stack frames state while entering the \texttt{fullversion()} function and also in a normal run of the program in the \texttt{chkserial()} function below:

\includegraphics[width=1\columnwidth]{5-full.png}
\includegraphics[width=1\columnwidth]{5-chk.png}

\textbf{6)} I have finally created the file \texttt{attack.ver} which has 16 random characters plus 8 more bytes to reach the pointer and finally adding the address of \texttt{rip} pointer of \texttt{fullversion()} to the end of file. The shell instruction to generate such file is brought in my Makefile as the \texttt{run} target. You can find the file below:

\includegraphics[width=1\columnwidth]{3-attack}

\section{Result}
You can see that I successfully attacked the program with a buffer overflow using making it call address of a different function, \texttt{fullversion()}. The result is as follows:

\includegraphics[width=1\columnwidth]{result.png}

%---------------------------------

\end{document}