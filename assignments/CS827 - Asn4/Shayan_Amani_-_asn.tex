
%---------------------------------
%	PACKAGES AND OTHER DOCUMENT CONFIGURATIONS
%---------------------------------

\documentclass[11pt]{article}

\input{structure.tex} % Include the file specifying the document structure and custom commands

%---------------------------------
%	ASSIGNMENT INFORMATION
%---------------------------------

\usepackage{inconsolata}

% Required
\newcommand{\assignmentQuestionName}{Question} % The word to be used as a prefix to question numbers; example alternatives: Problem, Exercise

\newcommand{\assignmentDueDate}{April, 20, 2020} % Due date
\newcommand{\assignmentClass}{CS 827} % Course/class
\newcommand{\assignmentTitle}{Assignment 4} % Assignment title or name
\newcommand{\assignmentAuthorName}{Shayan Amani (sa1149)} % Student name
%\newcommand{\assignmentClassInstructor}{PROFESSOR NAME} % Instructor name/time/description

%---------------------------------

\begin{document}

%---------------------------------
%	TITLE PAGE
%---------------------------------

\maketitle % Print the title page

\thispagestyle{empty} % Suppress headers and footers on the title page

\newpage

%---------------------------------
%	QUESTION 1
%---------------------------------

\begin{question}


\answer{
    These are the ways I came up with for Trudy to pretend:

    A replay attack to misuse a message for other purposes. Trudy can start two separate connections. From the first connection, she sends the first message that was sent out by Alice and then sends the message with R-1 to Bob from the second connection and then uses the response of the second connection to complete the first connection. The second connection leads to a time out but Bob is fooled into thinking that he is communicating with Alice on the first connection.

    Trudy can also record the messages being sent out by Alice and then send these messages to Bob. This way Trudy can keep sending the two messages that are being sent from Alice to Bob which would fool Bob into thinking that he is communicating with Alice.
    
}

\end{question}

%---------------------------------
%	QUESTION 2
%---------------------------------

\begin{question}

\questiontext{}

\answer{
    \textbf{a)} Yes, Bob is authenticated. To compute the encrypted message sent , Bob requires a S which is generated by Alice and encrypt with Bob's public key. By verifying the signature , Alice is confident that only Bob can have the required info and authenticated him.
    
    \textbf{b)} No, Bob does not send certificate request to Alice in message 2. This scenario requires Alice to have certificate for authentication. Alice does not provide its certification to Bob. So, Bob does not authenticate Alice.
}

\end{question}

%---------------------------------
%	QUESTION 3
%---------------------------------

\begin{question}

\answer{
    If Trudy comes in between as the man-in-the-middle, he can listen. $g^{ab}$ = $g^a^b$ and vice versa. Trudy shares secret $g^{at}$ mod p w/ Alice and similarly with Bob. They do authenticate each other and don't know that Trudy exits. And Trudy knows A's and B's keys. Trudy is able to drop messages between A and B.
}

\end{question}



%---------------------------------
%	QUESTION 4
%---------------------------------

\begin{question}
\answer{
    \textbf{1.} Alice's internet browser will check and catch the mistrusted certificate and throws a warning that the certificate verification has failed. Then Alice has option to whether stop the communication which leads to attack failure or continue at her own peril.

    \textbf{2.} The attack will succeed, if Alice chooses to continue at her discretion although the warning explicitly have mentioned the insecurity of the connection.
}
\end{question}

%---------------------------------
%	QUESTION 5
%---------------------------------

\begin{question}

\answer{
   \textbf{1.)} TICKET-GRANTING TICKET (TGT) is encrypted data that is issued by a server in order to initiate the authentication which give access to the network by tickets.

    \textbf{2.)} Alice's ticket includes Alice's keys and an expiration time and it is only valid  until its expiration time. In this way another security measure is added to the process which enhance the security.
    
    
    \textbf{3.} Only KDC can decrypt it so it will be decryptable by KDC. Accordingly, no other entity can change or access to the content. So KDC as a main entity in this process have the sole ability to decrypt which ultimately increase the security.
    
}


\end{question}

%---------------------------------

\end{document}